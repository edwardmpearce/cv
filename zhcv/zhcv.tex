%%%%%%%%%%%%%%%%%%%%%%%%%%%%%%%%%%%%%%%%%
% Medium Length Professional Chinese CV
% LaTeX Template
% Version 2.0 (8/5/13)
%
% This template has been downloaded from:
% http://www.LaTeXTemplates.com
%
% Original author:
% Trey Hunner (http://www.treyhunner.com/, https://github.com/treyhunner/resume)
%
% Modified (19/03/2020) to support Chinese language and profile picture by:
% Edward Pearce (https://edwardmpearce.github.io)
%
% Important note:
% This template requires the zhresume.cls file to be in the same directory as the
% .tex file. The zhresume.cls file provides the resume style used for structuring the
% document.
%
%%%%%%%%%%%%%%%%%%%%%%%%%%%%%%%%%%%%%%%%%

%----------------------------------------------------------------------------------------
%	PACKAGES AND OTHER DOCUMENT CONFIGURATIONS
%----------------------------------------------------------------------------------------

\documentclass{zhresume} % Use the custom zhresume.cls style

\usepackage[left=0.75in,top=0.6in,right=0.75in,bottom=0.6in]{geometry} % Document margins
\usepackage{hyperref}

\name{罗毅华 Edward Pearce} % Your name
\setimgpath{profilepic.jpg} % Path to a profile picture to be placed in the top right corner
\address{ +44(0)7783436944 \\ \href{edwardm.pearce@yahoo.co.uk}{edwardm.pearce@yahoo.co.uk} } % Contact details line 1 (e.g. phone and email)
\address{个人网站: \href{https://edwardmpearce.github.io}{edwardmpearce.github.io} \\ \href{https://www.linkedin.com/in/edwardm-pearce/}{linkedin.com/in/edwardm-pearce}} % Contact details line 2 (e.g. address)

\begin{document}

%----------------------------------------------------------------------------------------
%	EDUCATION SECTION
%----------------------------------------------------------------------------------------

\begin{rSection}{教育经历}

\begin{rSubsection}{谢菲尔德大学}{2016年10月 - 至今}{数学 全奖博士在读}{英国谢菲尔德}
\item 研究方向:代数几何与组合数学之间的关系
\item 主要课程: 使用图形处理器进行并行计算、交换代数、复微分几何
\item 教学:负责数学和统计学学院、工程学院本科及研究生的教学工作,教授课程涵盖基础核心数学、初中级计算机编程(Python, LaTeX, 和HTML)、R语言编程
\end{rSubsection}

%------------------------------------------------

\begin{rSubsection}{华威大学}{2012年10月 - 2016年7月}{数学 本硕连读}{英国考文垂}
\item 获得一等荣誉学士学位,均分92.8 (专业前2\%)
\item 教学:负责数学系本科一年级学生的小组研讨会,批改并讲解学生的作业
\end{rSubsection}

\end{rSection}

%----------------------------------------------------------------------------------------
%	WORK EXPERIENCE SECTION
%----------------------------------------------------------------------------------------

\begin{rSection}{实习经历}

\begin{rSubsection}{Health Data Insight}{2019年7月 - 2019年9月}{数据分析师}{英国剑桥}
\item 使用Python和SQL进行英国癌症病患的数据提取,使数据测试自动化、可视化
\item 建立框架分析患者肿瘤记录的综合数据集
\item 在公司年会向英国公共卫生部董事会成员展示汇报研究成果
\end{rSubsection}

%------------------------------------------------

\begin{rSubsection}{TAS Services Ltd}{2016年7月 - 2016年9月}{研究分析员}{中国香港}
\item 使用Python软件进行网页抓取、数据分析
\item 通过SQL和mongoDB软件将数据分析成果输出、倒入数据库
\item 编写数据自动化分析程序,从而高效对比不同信息源数据,进行精准的数据管理
\item 为统计建模编辑数据、链接数据库
\end{rSubsection}

\end{rSection}

%------------------------------------------------

\begin{rSection}{社团和组织经历}

\begin{rSubsection}{华威大学数学社团}{2014年2月 – 2016年4月}{主席}{华威大学}
\item 管理15人的社团管理层,主持社团会议,组织社团活动,管理900名社团会员
\item 成功获得两名新的社团赞助商,社团赞助金较前一年增加至二倍
\item 在华威大学、帝国理工大学联合本科会议上作暑期研究的演讲
\item 组织200名同年级数学系学生做线性代数期末考试复习讲座,并针对复习内容进行讲解
\end{rSubsection}

\end{rSection}

%----------------------------------------------------------------------------------------
%	TECHNICAL STRENGTHS SECTION
%----------------------------------------------------------------------------------------

\begin{rSection}{技能及其他}

\begin{tabular}{ @{} >{\bfseries}l @{\hspace{6ex}} l }
编程软件(精通) & Python、SageMath、C 语言 (OpenMP、 CUDA)、R语言 \\
编程软件(熟练) & Java、Matlab、Mathematica、Magma \\
数据库 & MySQL、Oracle SQL、mongoDB \\
其他计算机软件 & LaTeX、git、Unix shell (bash)、Hugo \\
语言 & 英文(母语)、中文(流利)、法语(基础) \\
兴趣爱好 & 电影、戏剧(莎士比亚)、 瑜伽、徒步
\end{tabular}

\end{rSection}

%----------------------------------------------------------------------------------------
%	EXAMPLE SECTION
%----------------------------------------------------------------------------------------

%\begin{rSection}{Section Name}

%Section content\ldots

%\end{rSection}

%----------------------------------------------------------------------------------------

\end{document}
